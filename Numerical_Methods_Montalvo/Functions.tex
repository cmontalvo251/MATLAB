\subsection{Functions - Palm - Chapter 3}

\begin{enumerate}

\item {\bf Create a Function}

You have already used a number of functions. When you use num2str,
linspace or inv. These are all built-in MATLAB functions that take in
a number of arguments and then outputs the result to the workspace. If
is possible to create functions that do whatever you wish. For
example, the following function computes the area of a circle.

\begin{framed}

function out = areac(in)

out = in\textrm{\^}2*pi;

\end{framed}

The first line of the code is the function header. The variable out is
the output of the function and areac is the name of the function and
finally in is the name of the input variable. If we try and run this
function from the editor MATLAB will throw an error. This is because
the input variable in is undefined. You must run the function from the
command window or a script

$>>$ areac(5)

will compute the area of a circle with a radius of 5.

\item {\bf Nested Functions}

It is possible to call a function within a function. The function
below will use the function areac to compute the volume of a cylinder.

\begin{framed}

function out = areacyl(radius,height)

out = areac(radius)*height;

\end{framed}

To call this function we then simply type

$>>$ areacyl(5,4)

which will compute the volume of a cylinder with a radius of 5 and a
height of 4. Note that the following is also acceptable as well.

$>>$ r = 5;

$>>$ h = 4;

$>>$ V = areacyl(r,h);

This will save the volume of the same cylinder into a variable V. 

\item {\bf Multiple Outputs}

Finally it is possible to output multiple values. The function below
computes the area of a circle and the volume of a cylinder.

\begin{framed}

function [Ac,Vcyl] = areacyl(radius,height)

Ac = areac(radius);

Vcyl = Ac*height;

\end{framed}

To call this you would need to use the brackets to accept two
variables.

$>>$ [A,V] = areacyl(r,h);

\end{enumerate}

