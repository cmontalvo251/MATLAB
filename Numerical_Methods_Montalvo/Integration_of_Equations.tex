\subsection{Integration of Equations - Chapter 22}

\begin{enumerate}

\item {\bf Richardson's Extrapolation}

  Recall that the error from the trapezoidal rule is given by
  
  \begin{equation}
    E_T = \frac{-1}{12 N^2} \bar{f}''(B-A)^3
  \end{equation}

  We know that moving to a higher order method can increase the
  accuracy. However because of the law of diminishing returns it is
  not always practical to move to a higher order method. Thus it is
  useful to write the actual computed value as a ratio from 1
  increment to another. That is it is possible to write

  \begin{equation}
    I = I(N_1) + E(N_1)
  \end{equation}

  where $I(N_1)$ is the solution of the trapezoidal rule for $N_1$
  intervals. Because of the method certain errors are accrued as we've
  seen and is given as $E(N_1)$. Now it is equally valid to assume that
  $N_2 = 2N_1$ then

  \begin{equation}
    I = I(N_2) + E(N_2)
  \end{equation}

  Thus, $I(N_1) + E(N_1) = I(N_2) + E(N_2)$. The ratio of errors can
  then be written as

  \begin{equation}
    \frac{E(N_1)}{E(N_2)} = \left(\frac{N_2}{N_1}\right)^2
  \end{equation}

  substituting this into our two equations of I and solving for
  $E(N_2)$ we have

  \begin{equation}
    E(N_2) = \frac{I(N_1)-I(N_2)}{1-\left(\frac{N_2}{N_1}\right)^2}
  \end{equation}

  we can then plug this into the equation involving $N_2$ and collect
  terms to arrive at

  \begin{equation}
    I = \frac{4}3 I(N_2) - \frac{1}3 I(N_1)
  \end{equation}

  Remember that $N_2 = 2N_1$. The result of this derivation is that is
  possible to compute $I(N_2)$ and $I(N_1)$ and obtain a more
  accurate result of I. It can be shown that if $I(N_1)$ is of
  $O(\Delta x^2)$, I is $O(\Delta x^4)$. Essentially you have
  increased the order of the method without actually using another
  method.

  \item {\bf Romberg Integration}

    Romberg Interpolation is simply an extension of Richardson's
    Extrapolation. 

    \begin{equation}
      I_{j,k+1} = \frac{4^k I_{j+1,k} - I_{j,k}}{4^k - 1}
    \end{equation}

    j is the interval number and k is the order of the method. The
    trapezoidal method is of order 1 thus k = 1. When k = 1 and j = 1, Romberg
    integration reduces to 

    \begin{equation}
      I_{1,2} = \frac{4 I_{2,1} - I_{1,1}}{4 - 1}
    \end{equation}

    Here $I_{2,1}$ is $I(N_2)$ using the trapezoidal rule. $I_{1,1}$
    is $I(N_1)$ using the trapezoidal rule. $I_{1,2}$ is the
    equivalent of $I(N_1)$ using a second order method. The power of
    Romberg Integration is that it does not stop there. It is possible
    to increase k and j until you get the estimate you like. For
    example, the table below shows the flow of Romberg Integration to
    compute $I_{1,4}$.

    \begin{center}
      \begin{tabular}{c | c c c c c c c}
        Interval & First Order & & Second Order & & Third Order & & Fourth
        Order \\
        \hline
        $N_1$ & $I_{1,1}$ & $\longrightarrow$ & $I_{1,2}$ & $\longrightarrow$ &
        $I_{1,3}$ & $\longrightarrow$ & $I_{1,4}$ \\
        $N_2=2N_1$ & $I_{2,1}$ & $\nearrow$ & $I_{2,2}$ & $\nearrow$ &
        $I_{2,3}$ &  $\nearrow$ \\
        $N_3=2N_2$ & $I_{3,1}$ & $\longrightarrow$ & $I_{3,2}$ & $\nearrow$ &
        &  \\
        $N_4=2N_3$ & $I_{4,1}$ & $\nearrow$ & & &     &  \\
      \end{tabular}
    \end{center}

    This shows that it is possible to obtain accuracy on the order of
    $\Delta x^8$ by only using methods that have accuracy on the order
    of $\Delta x^2$.

\item {\bf Gauss Quadrature}

\end{enumerate}
