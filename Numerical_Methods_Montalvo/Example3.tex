\subsection{Example Problems}

1.) Using rules of mathematical precedence, solve for the following
with MATLAB (writing it exactly as it is printed). Compare answers by
hand. Did you get the same answer? 

\begin{center}
\begin{tabular}{l c r}
a) 81\textrm{\^}(3/4)+5*2\textrm{\^}2/2-5 & b) 12*4/3-8\textrm{\^}2/2-15/(5-2)
\end{tabular}
\end{center}

2.) Suppose that x=4, and y=3. Use MATLAB to compute the
following. Check your answers by hand.

\begin{equation} \nonumber
\begin{matrix}
a) \frac{yx^3}{x-y} & b) \frac{x^5}{y^5-1} & c) 2\pi x^2y & d) \frac{4 \sqrt{y-2}}{3x-5}
\end{matrix}
\end{equation}

3.) Evaluate the following expressions in MATLAB for the given value
of x. Check your work by hand.

\begin{equation} \nonumber
\begin{matrix}
a) y=5x^3 & x=3 & b) y=2\frac{sin x}7 & x=15^0 & c) y=6x^{1/3}+\frac{2x}3 & x=27
\end{matrix}
\end{equation}

4.) Suppose that x=-6-5i and y=4+2i. Use MATLAB to compute the
following. Check your work by hand.

\begin{equation} \nonumber
\begin{matrix}
a) x + y & b) x/y & c) xy
\end{matrix}
\end{equation}

5.) Use MATLAB to calculate the following, and check your answers with
a calculator.

\begin{equation} \nonumber
\begin{matrix}
a) e^{{-2.3}^2}+2.86log_{10}(14) + \sqrt[6]{516} & b) \frac{5~ln 7}2 +
\sqrt{5^2+4^3} & c) cos(\frac{3\pi}8)(sinh(\frac{3\pi}4))^2
\end{matrix}
\end{equation}

6.) The Richter scale is a measure of the intensity of an
earthquake. The energy E (in Joules) released by the quake is related
to the magnitude M on the Ricther scale as follows:

\begin{equation} \nonumber
E = 10^{4.4}10^{1.5M}
\end{equation}

Use MATLAB to determine how much more energy is released by a magnitude 7.5 quake than a 6.1
quake. 
\ \\

7.) Convert the following strings to numbers using the str2num()
function. What happens?

\begin{center}
\begin{tabular}{l c r}
a) '42' & b) 'hello' & c) 'h' \\ d) '4.2' & e) '4/2' & f) 'cos(pi)'
\end{tabular}
\end{center}

8.) Convert the following numbers to strings using num2str()
function. What happens? Using the length function compute the length of
each string.

\begin{center}
\begin{tabular}{l c r}
a) 42 & b) -15 & c) exp(1) \\ d) 4.2 & e) 4/2 & f) cos(pi)
\end{tabular}
\end{center}

9.) Using the function exist() test whether or not these variables
exist. Note depending on how you save your workspace you may get
different answers however verify the output using whos.

\begin{center}
\begin{tabular}{l c r}
a) x & b) y & c) z
\end{tabular}
\end{center}

If they do exist using the disp() function print the following to the command
window:
\ \\

Yes the variable \_\_\_\_\_ exists.
\ \\

10.) Type this matrix in MATLAB.
\ \\
\begin{equation} \nonumber
A = \begin{bmatrix} 2 & 7 & -3 & 9 \\ -3 & 5 & 15 & 3 \\ 4 & 11 & 8 &
  13 \\ 16 & 4 & -5 & -11
  \\ 5 &-2 &18 &3 \end{bmatrix}
\end{equation}

Use MATLAB to answer the following
questions. 
\ \\

a) Create a vector consisting of the elements in the third column of
{\bf A}.

b) Create a vector consisting of the elements in the second row of
{\bf A}

c) Create a submatrix encompassing the lower right 3x3 matrix only.

d) Find the value of $a_{32}$.
\ \\

11.) Given the matrices

\begin{equation} \nonumber
\begin{matrix}
A = \begin{bmatrix} 3 & 2 & 4 \\ 1 & 5 & -3 \\ 4 & -10 & 0\end{bmatrix} &
B = \begin{bmatrix} 11 & 0 & -3 \\ 5 & -12 & 4 \\ 2 & 3 & 1\end{bmatrix} &
C = \begin{bmatrix} 7 & 15 & 1 \\ 10 & 3 & -2 \\ 9 & -5 & 8\end{bmatrix} &
\end{matrix}
\end{equation}

Compute the following. Check your work by hand.
\ \\

a) Compute {\bf A}+{\bf B}+{\bf C}

b) Compute {\bf A}-{\bf B}-{\bf C}

c) Does 5({\bf A} + {\bf C}) = 5{\bf A} + 5{\bf C}?

d) Does {\bf A}*{\bf C} = {\bf C}*{\bf A}?

e) Does ({\bf A}+{\bf B})$^T$ = {\bf A}$^T$ + {\bf B}$^T$?

f) Does {\bf A}/{\bf B} = {\bf A}./{\bf B}?
\ \\

12.) Use matrices to solve for x,y and z.

\begin{equation} \nonumber
\begin{matrix}
5x-3y+4z = 41 \\
12x + 6y -7z = -26 \\
-4x + 2y + 6z = 14 
\end{matrix}
\end{equation}
\ \\

Check your answer by hand.
\ \\

13.) Plot the equation below. Use the bounds [-2 2]. Label your
axes, make a title, turn on the grid and make the background of your
plot white. Plot a circle where the graph crosses the x axis. 

\begin{equation} \nonumber
y = 5x^3 + 2x^2 - 5x - 20
\end{equation} 

14.) Plot the equation from above. Use the bounds [-2 2]. Label your
axes, make a title, turn on the grid and make the background of your
plot white. Plot a circle where the graph crosses the x axis. Is this
the same as your answer in problem 13?
\ \\

15.) Assume that an aircraft is flying at a certain speed (mi/hr) for
4 different legs. Assume that this speed is constant for a certain
amount of time as given by the table below.
\ \\

\begin{tabular}{l c c c r}
Leg & 1 & 2 & 3 & 4 \\
Speed (mi/hr) & 200 & 250 & 400 & 300 \\
Time (hr) & 2 & 5 & 2.5 & 1.2
\end{tabular}
\ \\

a.) Input all the data into a cell array including the table
header. For example, the first row should read:
\ \\

aircraft\_data = \{'Leg',1,2,3,4\};
\ \\

b.) Compute the distance traveled by the aircraft in each leg using
the cell array you defined in part a.

d.) Input all data into a structure assuming the
following fields: 'Leg', 'Speed\_mph' and 'Time\_hr'. 

e.) Compute the distance traveled and add a field to the structure
from part d called 'Distance\_mi'
\ \\
