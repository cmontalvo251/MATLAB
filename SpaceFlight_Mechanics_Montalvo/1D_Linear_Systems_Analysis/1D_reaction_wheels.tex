%\documentclass[]{aiaa-tc} %load in aiaa class template file
\documentclass{article}
\usepackage{amsmath}
\usepackage{mathrsfs}
\usepackage{enumitem}
\usepackage{mathabx}
\usepackage{hyperref}
\usepackage{natbib} %REMOVE THIS IF YOU DON'T WANT BRACKETS ON YOUR PAPER
\usepackage{float}
\usepackage{color,soul}
\usepackage[font=normalsize,labelfont=bf]{caption}
\usepackage[left=2cm,right=2cm,top=2cm,bottom=2cm]{geometry}
\usepackage{graphicx}

\newcommand{\sref}[1]{$^{[\ref{#1}]}$}
\newcommand{\dref}[2]{$^{[\ref{#1}-\ref{#2}]}$}

 % Define commands to assure consistent treatment throughout document
\newcommand{\eqnref}[1]{(\ref{#1})}
\newcommand{\class}[1]{\texttt{#1}}
\newcommand{\package}[1]{\texttt{#1}}
\newcommand{\file}[1]{\texttt{#1}}
\newcommand{\BibTeX}{\textsc{Bib}\TeX}

\begin{document}

\linespread{1}

\newpage

\begin{center}
  {\LARGE CubeSAT Attitude Control}
\end{center}

\begin{abstract}

A CubeSAT is a small satellite on the order of 10 centimeters along
each axis. A 1U satellite is a small cube with 10 cm sides. These
satellites are used for a variety of missions and created by a variety
of different organizations. When deployed from 
a rocket, a CubeSAT may obtain a large angular velocity
which must be reduced before most science missions or communications
can take place. Maximizing solar energy charging also involves
better pointing accuracy. To control the attitude of these small
satellites, reaction wheels, magnetorquers and even the gravity
gradient are used in low earth orbit (LEO) while reaction control thrusters
are typically used in deep space. On a standard LEO CubeSAT, 3 reaction
wheels are used as well as 3 magnetorquers. In the initial phase of
the CubeSAT mission, the magnetorquers are used to reduce the angular
velocity of the satellite down to a manageable level. Once the norm of the angular velocity is
low enough, the reaction wheels can spin up reducing the angular
velocity to zero. This paper investigates the necessary
mathematics to understand the control architecture necessary for a
simple CubeSAT.

\end{abstract}

%%%%%%%%%%%%%%%%%%%%%%%%%%%%%%%%%%%%%%%%%%%%%%%%%%%%%%%%%%%%%%%%%%%
%%%%%%%%%%%%%%%%%%%%%%%%%%%%%%%%%%%%%%%%%%%%%%%%%%%%%%%%%%%%%%%%%%%
%%%%%%%%%%%%%%%%%%%%%%%%%%%%%%%%%%%%%%%%%%%%%%%%%%%%%%%%%%%%%%%%%%%

\begin{tabbing}
  XXXXXXXXXX \= \kill% this line sets tab stop
  $x,y,z$ \> components of the mass center position vector in the
  inertial frame (m)  \\
  $\phi,\theta,\psi$ \> Euler roll,pitch, and yaw angles (rad) \\
  $q_0,q_1,q_2,q_3$ \>  quaternions \\  
  $u,v,w$ \> components of the mass center velocity vector in the
  body frame (m/s)  \\
  $p,q,r$ \> components of the mass center angular velocity vector in the
  body frame (rad/s)  \\
  $\vec{\omega}_{B/I}$ \> angular velocity vector of the satellite in
  the body frame (rad/s) \\
  ${\bf T}_{IB}$ \> rotation matrix from frame I to frame B \\
  ${\bf H}$ \> relationship between angular velocity components in
  body frame and derivative of Euler angles \\
  $m$ \> mass (kg) \\
  $I$ \> mass moment inertia matrix about the mass center
  in the body frame ($kg-m^2$)  \\
  $X,Y,Z$ \> components of the total force applied to CubeSAT in
  body frame (N)  \\
  $L,M,N$ \> components of the total moment applied to CubeSAT in
  body frame (N-m)  \\
  ${\vec r}_{A\rightarrow B}$ \> position vector from a generic point A
  to a generic point B (m) \\
  ${\vec V}_{A/B}$ \> velocity vector of a generic point A
  with respect to a generic frame B (m/s) \\
  ${\bf S}(\vec{r})$ \> skew symmetric matrix operator on a
  vector. Multiplying this matrix by a vector is equivalent \\
  \> to a cross product\\
\end{tabbing}

%%%%%%%%%%%%%%%%%%%%%%%%%%%%%%%%%%%%%%%%%%%%%%%%%%%%%%%%%%%%%%%%%%%
%%%%%%%%%%%%%%%%%%%%%%%%%%%%%%%%%%%%%%%%%%%%%%%%%%%%%%%%%%%%%%%%%%%
%%%%%%%%%%%%%%%%%%%%%%%%%%%%%%%%%%%%%%%%%%%%%%%%%%%%%%%%%%%%%%%%%%%

\section{Reaction Wheel Model}

The reaction wheel model must be included before the attitude dynamics
beause they directly affect the inertia of the satellite. There are
three reaction wheels on this satellite but to simplify the dynamics a
1D system will be used. Thus 1 reaction wheel will be modeled. This
reaction wheel has an angular velocity $\dot{\theta}_{R}$ and angular acceleration
$\ddot{\theta}_{R}$. The inertia of the reaction wheel is first written
about the center of mass of the reaction wheel and is given by the
equation below where the reaction wheel is modeled as a disk with
finite radius $(r_{RW})$ and height $(h_{RW})$. The subscript $R$ is used to
denote that this inertia matrix is about the center of mass of the
reaction wheel while the super script $R$ is used to denote the frame
of reference in this case the body frame of the CubeSAT 
\begin{equation}
  I^{B}_{R} = m_{R}r^2/2
\end{equation}
The parallel axis theorem can then be used to shift the inertias to
the center of mass of the satellite where the subscript $RB$ denotes
the reaction wheel inertia taken about the center of mass of the
satellite. 
\begin{equation}
  I^{B}_{RB} = I^{B}_{R} + m_{R}(x_R^2+y_R^2)
\end{equation}
Where $x_R,y_R$ are the distances from the center of mass of
the satellite to the center of mass of the reaction wheel in the
satellite body reference frame. The total inertia of the entire
satellite-reaction wheel system is then just a sum of the satellite
and the reaction wheels
\begin{equation}
   I_S = I_B + I^{B}_{RB}
\end{equation}
The total angular momentum of the satellite is then equal to the following
equation where $\dot{\theta}$ is the angular velocity of the
satellite. 
\begin{equation}
  H_S = I_B\dot{\theta} + I^{B}_{RB}(\dot{\theta}_{R}+\dot{\theta})
\end{equation}
In a similar fashion, the total torque placed on the satellite is
equal to the following
\begin{equation}
  M_{R} = I^{B}_{RB}\ddot{\theta}_{R}
\end{equation}
The reaction wheel acceleration is controlled by an input torque from
a motor. However the input torque is not constant due to the fact that
the reaction wheel cannot apply anymore torque once the reaction wheel
has reached its maximum angular velocity. Thus the equations of motion
for the reaction wheel are given by
\begin{equation}
  \ddot{\theta}_{R} = T/I^{B}_{RB}
\end{equation}
where $|T_{MAX}| = |T_0 -
\frac{T_0}{\dot{\theta}_{R,max}}\dot{\theta}_R|$. 

\section{Rotational Equations of Motion in 1D}

The equations of motion of a satellite in free space can be given
using the equation below assuming the inertia matrix is constant. 
\begin{equation}
M_R = I_S\ddot{\theta}
\end{equation}

\section{Control Schemes for Reaction Wheels}

Assuming each reaction is aligned with a principal axis of inertia the
control scheme is extremely simple. The derivation here will just be
for the aligned case. In this analysis it is assumed that a torque can
be applied to the reaction wheel and thus the angular acceleration of the
reaction wheel can be directly controlled through the torque input
$T$. Assuming this, a simple PD 
control law can be used to orient the satellite at any desired
orientation. 
\begin{equation}
  T_C = -k_p(\tilde{\theta}-\theta_{C})-k_d(\tilde{\dot{\theta}}-\dot{\theta}_{C})
\end{equation}
Note that $T_C$ is used here since the system cannot
generate torque from nowhere. As such the torque must first pass
through a first order filter.
\begin{equation}
  \dot{T} + a T = T_C
\end{equation}
where $a$ is the cutoff frequency of the first order
filter. Furthermore $\tilde{\theta}$ is used since the sensor is not
perfect or immediate and also introduces some delay such that
\begin{equation}
  \dot{\tilde{\theta}} + b \tilde{\theta} = \theta
\end{equation}
where $b$ is the cutoff frequency of the sensor. In order
to design and select reaction wheels the maximum angular 
momentum of the satellite must be obtained by assuming the worst case angular
velocity times the moment of inertia of the satellite.
\begin{equation}
  H_{required} = n||I_s\dot{\theta}_{MAX}||
\end{equation}
With this reaction wheels can be selected based on this worst case
scenario plus a safety factor of $n$ which is typcially set to 2 in
spacecraft operations. If reaction wheels cannot be used in the event
of saturation or other issues reaction control thrusters can be
used. Typically a value of $|\dot{\theta}_{MAX}|=5^o/sec$ is
used. 

\end{document}
